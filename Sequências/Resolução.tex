\documentclass[11pt]{article}

\usepackage[svgnames]{xcolor}
\usepackage{tcolorbox}
\usepackage[utf8]{inputenc}
\usepackage{lmodern}
\usepackage{indentfirst}
\usepackage{microtype}
\usepackage{amsmath}
\usepackage[brazil]{babel}
\usepackage{parskip}
\usepackage[tmargin=1.5in, lmargin=1.25in, rmargin=1.25in, bmargin=1.0in]{geometry}

\title{Resolução da L.E (P.A e P.G)}
\author{Mickael Lima}
\date{Dezembro, 2021}

\begin{document}

\maketitle

\section{Progressão Aritmética}

\subsection{Fundamentos}

\subsubsection{Questão 1}

A questão lida com a formação básica de uma P.A. Para determinar o termo $x$ pedido, podemos igualar a constante de razão $r$ da seguinte forma:

\begin{tcolorbox}[colback=LightYellow]
\[(2x + 1) - x = (5x + 7) - (2x + 1)\]
\[x + 1 = 3x + 6\]
\[-2x = 5\]
\[x = -\frac{5}{2}\]
\end{tcolorbox}

A prova real é realizada substituíndo $x$ na sequência e verificando se de fato a mesma forma uma P.A com os três primeiros termos dados.

\begin{tcolorbox}[colback=LightYellow]
\[f = \left(-\frac{5}{2}, -2\cdot \frac{5}{2} + 1, -5\cdot \frac{5}{2} + 7 \right)\]
\[f = \left(-2.5, -4, -5.5\right)\]
\end{tcolorbox}

De fato, forma-se a P.A de $a_{1} = (-2.5)$ e $r = (-1.5)$.

\subsubsection{Questão 2}

O enunciado pede duas condições a ser cumpridas (além da formação em P.A)

\begin{itemize}
  \item A soma dos 3 números deve ser igual a 3
  \item A soma dos 3 números, cada um ao quadrado, deve ser igual a 11
\end{itemize}

Isso implica no seguinte sistema

\begin{tcolorbox}[colback=LightYellow]
\begin{equation*}
\begin{cases}
  (x - r) + x + (x + r) = 3 \\
  (x - r)^2 + x^2 + (x + r)^2 = 11 \\
\end{cases}
\end{equation*}
\end{tcolorbox}

Do primeiro membro, nota-se inicialmente que $r$ irá ser cancelado pela soma entre $-r$ e $+r$. Ficando apenas $3x = 3$. Desse modo, descobre-se $x = 1$. Jogando essa informação no segundo membro, poderemos manipular e descobrir $r$ efetivamente.

\begin{tcolorbox}[colback=LightYellow]
\[(1 - r)^{2} + 1 + (1 + r)^{2} = 11\]
\[(1^{2} - 2r + r^{2}) + 1 + (1^{2} + 2r + r^{2}) = 11\]
\end{tcolorbox}

\begin{tcolorbox}[colback=LightYellow]
\[3 + 2r^{2} = 11 \]
\[r = \pm 2\]
\end{tcolorbox}

Sendo $x = 1$ e $r = \pm 2$, podemos montar duas progressões aritméticas $f$ (para $r = 2$) e $g$ (para $r = -2$), como ilustrado abaixo.


\begin{tcolorbox}[colback=LightYellow]
  \[f = (-1, 1, 3)\]
  \[g = (3, 1, -1)\]

  \begin{itemize}
          \item Soma: $-1 + 1 + 3 = 3$
          \item Soma dos quadrados: $3^{2} + 1^{2} + (-1)^{2} = 11$
  \end{itemize}
\end{tcolorbox}

\subsubsection{Questão 3}

Chamaremos $l$ de lado, $d$ a diagonal e $a$ a área. Montemos a sequência $f = (l, d, a)$. Como o enunciado pede o valor algébrico de $l$, é interessante reescrever todos esses valores em função do mesmo. Utilizando os conhecimentos importados da Geometria Plana, podemos afirmar que a P.A terá a forma

\begin{tcolorbox}[colback=LightYellow]
\[f = (l, l\sqrt{2}, l^{2})\]
\end{tcolorbox}

A P.A se estabelecerá com a existência de $r$. Utilizando a mesma técnica da primeira questão, temos

\begin{tcolorbox}[colback=LightYellow]
\[l\sqrt{2} - l = l^{2} - l\sqrt{2} \]
\[l\sqrt{2} + l\sqrt{2} - l = l^{2}\]
\[2\cdot l\sqrt{2} - l = l^{2}\]
\[l = \frac{2\cdot l\sqrt{2} - l}{l}\]
\[l = \frac{2\cdot l\sqrt{2}}{l} - \frac{l}{l}\]
\[l = 2\sqrt{2} - 1\]
\end{tcolorbox}

\end{document}
