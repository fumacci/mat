\documentclass[11pt]{article}

\usepackage[svgnames]{xcolor}
\usepackage{tcolorbox}
\usepackage[utf8]{inputenc}
\usepackage{lmodern}
\usepackage{indentfirst}
\usepackage{microtype}
\usepackage{amsmath}
\usepackage[brazil]{babel}
\usepackage{parskip}
\usepackage[tmargin=1.5in, lmargin=1.25in, rmargin=1.25in, bmargin=1.0in]{geometry}

\title{Sequências e Progressões II}
\author{Mickael Lima}
\date{Dezembro, 2021}

\begin{document}

\maketitle
\pagebreak
\tableofcontents
\pagebreak

\section{Progressão Geométrica}

A progressão geométrica é outro tipo de progressão, semelhante à aritmética. É definida pela fórmula de recorrência ilustrada por

\begin{equation*}
\begin{cases}
  a_{1} \\
  a_{n} = a_{n - 1} \cdot q
\end{cases}
\end{equation*}

Sendo $a, q$ números reais fornecidos (ou pelo menos implicito, no caso de $q$). Sendo assim, uma P.G é toda sequência a qual o próximo termo $a_{n}$ seja igual ao termo anterior $a_{n - 1}$ multiplicado por uma constante $q$ (que equivale ao $r$ da P.A).

\subsection{Classificação}

As P.Gs são classificadas de 5 modos diferentes.

\begin{tcolorbox}[colback=LightYellow]

\begin{itemize}
        \item Crescente: o próximo termo é maior que o anterior
        \begin{equation*}
        \begin{align*}
          a_{n} &> a_{n - 1} \\
          a_{n - 1} \cdot q &> a_{n - 1} \\
          q &> \frac{a_{n - 1}}{a_{n - 1}} \\
          q &> 1 \\
        \end{align*}
        \end{equation*}
\end{itemize}

Nesse caso, a P.G será crescente quando $q > 1$ \textbf{e somente para sequências positivas}. Para casos em que $a_{1}$ é negativo, vale a relação $0 < q < 1$.
\end{tcolorbox}

\begin{tcolorbox}[colback=LightYellow]
\begin{itemize}
  \item Constante: Há duas situações em que isso acontece.
        \begin{itemize}
                \item Quando $q = 1$ (já que não haverá variação na multiplicação de $a_{n}$ e $a_{n - 1}\cdot q$).
                \item Quando $a_{1} = 0$ (já que multiplicar por zero a deixará constante).
        \end{itemize}
\end{itemize}
\end{tcolorbox}

\begin{tcolorbox}[colback=LightYellow]
\begin{itemize}
  \item Decrescente: há dois casos para se analisar (semelhante à crescente).
        \begin{itemize}
                \item P.G positiva: será decrescente para $0 < q < 1$
                \item P.G negativa: será decrescente para $q > 1$
        \end{itemize}
\end{itemize}
\end{tcolorbox}

\begin{tcolorbox}[colback=LightYellow]
\begin{itemize}
  \item Alternantes: o próximo termo tem sinal contrário ao anterior. Isso ocorre sempre que $q < 0$, forçando a alternância de sinais.

  \item Estacionárias: Quando $q = 0$ e $a_{1} \neq 0$, forçando-a a ficar constante após o primeiro termo.
\end{itemize}
\end{tcolorbox}

\subsection{Notações Úteis}

Tal qual descrito nas notas sobre P.A, as notações úteis da P.G podem ser escritas (como exemplo, os 3 primeiros termos da P.G) como.

\[\left(x, x\cdot q, x\cdot q^{2}\right)\]
\[\left(\frac{x}{q}, x, x\cdot q\right)\]

Para 4 termos, têm-se $\left(x, xq, xq^{2}, xq^{3}\right)$, para $n$ termos, $\left(x, xq, xq^{2}, \dots, xq^{n - 1}\right)$

\begin{tcolorbox}[colback=LightYellow]
  \begin{itemize}
    \item Qual número deverá ser somado a 1, 9 e 15 para termos, nessa ordem, três números em P.G.
  \end{itemize}
\end{tcolorbox}

\begin{tcolorbox}[colback=LightYellow]
  A sequência em P.G $f$ terá a forma de $f = \{(1 + x), (9 + x), (15 + x)\}$. O próximo termo deverá ser igual ao produto entre o termo anterior com uma constante $q$. Para que essa constante $q$ exista, é estabelecida a seguinte relação.

  \[q = \frac{9 + x}{1 + x} = \frac{15 + x}{9 + x}\]

  Portanto, constrói-se a seguinte expressão

  \[(9 + x)^{2} = (15 + x)(1 + x)\]
  \[81 + 18x + x^{2} = 15 + 15x + x + x^{2}\]
  \[81 + 18x = 15 + 16x\]
  \[2x = -66\]
  \[x = -33\]
\end{tcolorbox}

\subsection{Fórmula do enésimo termo}

Semelhante à progressão aritmética, a P.G pode ser armada do termo $a_{1}$ até $a_{n}$ ($a_{1} \neq 0$, $q \neq 0$, $n$ conhecido) da seguinte forma.

\begin{equation*}
\begin{cases}
  a_{2} = a_{1}\cdot q \\
  a_{3} = a_{2}\cdot q \\
  \dots \\
  a_{n} = a_{n - 1} \cdot q
\end{cases}
\end{equation*}

Caso o primeiro lado ($a_{2}, \dots$) seja multiplicado, e o segundo lado $(a_{1}\cdot q), (a_{2}\cdot q), \dots$ também, é evidente que a igualdade se manterá, formando a seguinte equação.

\[a_{2} \cdot a_{3}\dots a_{n} = (a_1\cdot q) \cdot (a_2\cdot q) \cdot (a_3\cdot q) \dots (a_{n - 1} \cdot q)\]

É possível pôr o $q$ em evidência, visto que ele aparece $n - 1$ vezes ($a_{n}$ não é contado) no segundo membro.

\[[a_{2} \cdot a_{3}\dots, a_{n - 1}]\cdot a_{n} = (a_1 [\cdot a_2 \cdot a_3 \dots a_{n - 1}])\cdot q^{n - 1}\]

O que está destacado por colchetes se cancelam, formando a equação final em função de $n$.


\begin{tcolorbox}[colback=LightYellow]
  \[a_{n} = a_1\cdot q^{n - 1}\]
\end{tcolorbox}

\subsubsection{Demonstração por Indução Finita}

\begin{itemize}
  \item Checa-se a validade para $n = 1$

  \[a_{1} = a_{1}\cdot q^{1-1}\]
  \[a_{1} = a_{1}\]

  \item Admite-se válido para $n = p$
        \[a_{p} = a_{1}\cdot q^{p - 1}\]

  \item Checa-se a validade para $n = p + 1$

        \[a_{p + 1} = a_{1}\cdot q^{(p + 1) - 1} \Leftrightarrow a_{1}\cdot q^{p-1}\cdot q\]
\end{itemize}

\end{document}
