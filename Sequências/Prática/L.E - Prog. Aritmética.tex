\documentclass[12pt, addpoints]{exam}
\usepackage[brazil]{babel}

\pointpoints{ponto}{pontos}
\totalformat{Pergunta \thequestion: \totalpoints pontos}

\chqword{Pergunta}
\chpgword{Página}
\chpword{Pontos}
\chbpword{Pontos extras}
\chsword{Pontos obtidos}
\chtword{Total}
\begin{document}


\begin{center}
\fbox{\fbox{\parbox{5.5in}{\centering
Lista de Exercícios - Progressão Aritmética}}}
\end{center}

\subsection*{Fundamentos}

\begin{questions}
  \question [\half] Determine $x$ de modo que $(x ; 2x + 1 ; 5x + 7)$ seja uma P.A.

  \question [\half] Obtenha 3 números em P.A, de modo que a soma entre esses seja igual a 3 e a soma de seus quadrados seja 11

  \question [\half] Os números que exprimem o lado, a diagonal e a área de um quadrado estão, respectivamente, em P.A. Determine quanto mede o lado.

  \question [1 \half] Demonstre que se $(a, b, c)$ é uma P.A, então $(a^{2}bc, ab^{2}c, abc^{2})$ também é.

  \question[1] Determine qual é o primeiro termo negativo da P.A $f = \{60, 53, 46, \dots\}$.

  \question [\half] Sabe-se que $a_{10} = 7$ e que $a_{12} = -8$. Monte a P.A.
  \question [\half] De 100 a 1000, quanto são os múltiplos de 2 ou 3.
  \question [1] Obtenha a soma dos 200 primeiros termos da sequência dos números ímpares positivos. Cálcule também a soma dos $n$ primeiros termos iniciais (par e ímpar).

  \question[1] Um jardineiro tem que regar 60 roseiras plantadas ao longo de uma vereda retilínea e distando 1 m uma da outra. Ele enche seu regador numa fonte situada na mesma vereda, a 15 m da primeira roseira, e a cada viagem rega 3 roseiras. Começando e terminando na fonte, determine qual é o percurso total que ele terá que caminhar até regar todas as roseiras.

  \question[3] \textbf{(ITA-SP)} Considere a progressão aritmética ($a_{1}, a_{2}, a_{3}, \dots, a_{50}$) de razão $d$. Se $\displaystyle \sum_{n = 1}^{10} a_{n} = 10 + 25d$ e $\displaystyle \sum_{n = 1}^{50} = 4550$. Então $d - a_{1}$ é igual a:
  \vspace{0.2cm}
  \begin{parts}
    \part 3
    \part 6
    \part 9
    \part 11
    \part 14
  \end{parts}


\end{questions}

\begin{center}
\gradetable[h][questions]
\end{center}

\begin{itemize}
  \item 7 ou acima: A revisão não é necessariamente de alta prioridade
  \item Entre 5 e 6.5: A revisão é de média prioridade
  \item Abaixo: A revisão fundamental é necessária e de alta prioridade
\end{itemize}

\end{document}
