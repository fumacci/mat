\section*{Resolução das Questões}
\subsection*{Progressão Aritmética}

\subsubsection*{Questão 1}

\hrulefill

A questão lida com a formação básica de uma P.A. Para determinar o termo $x$ pedido, podemos igualar a constante de razão $r$ da seguinte forma:

\begin{tcolorbox}[colback=LightYellow]
\[(2x + 1) - x = (5x + 7) - (2x + 1)\]
\[x + 1 = 3x + 6\]
\[-2x = 5\]
\[x = -\frac{5}{2}\]
\end{tcolorbox}

A prova real é realizada substituíndo $x$ na sequência e verificando se de fato a mesma forma uma P.A com os três primeiros termos dados.

\begin{tcolorbox}[colback=LightYellow]
\[f = \left(-\frac{5}{2}, -2\cdot \frac{5}{2} + 1, -5\cdot \frac{5}{2} + 7 \right)\]
\[f = \left(-2.5, -4, -5.5\right)\]
\end{tcolorbox}

De fato, forma-se a P.A de $a_{1} = (-2.5)$ e $r = (-1.5)$.

\subsubsection*{Questão 2}

\hrulefill

O enunciado pede duas condições a ser cumpridas (além da formação em P.A)

\begin{itemize}
  \item A soma dos 3 números deve ser igual a 3
  \item A soma dos 3 números, cada um ao quadrado, deve ser igual a 11
\end{itemize}

Isso implica no seguinte sistema

\begin{tcolorbox}[colback=LightYellow]
\begin{equation*}
\begin{cases}
  (x - r) + x + (x + r) = 3 \\
  (x - r)^2 + x^2 + (x + r)^2 = 11 \\
\end{cases}
\end{equation*}
\end{tcolorbox}

Do primeiro membro, nota-se inicialmente que $r$ irá ser cancelado pela soma entre $-r$ e $+r$. Ficando apenas $3x = 3$. Desse modo, descobre-se $x = 1$. Jogando essa informação no segundo membro, poderemos manipular e descobrir $r$ efetivamente.

\begin{tcolorbox}[colback=LightYellow]
\[(1 - r)^{2} + 1 + (1 + r)^{2} = 11\]
\[(1^{2} - 2r + r^{2}) + 1 + (1^{2} + 2r + r^{2}) = 11\]
\end{tcolorbox}

\begin{tcolorbox}[colback=LightYellow]
\[3 + 2r^{2} = 11 \]
\[r = \pm 2\]
\end{tcolorbox}

Sendo $x = 1$ e $r = \pm 2$, podemos montar duas progressões aritméticas $f$ (para $r = 2$) e $g$ (para $r = -2$), como ilustrado abaixo.

\begin{tcolorbox}[colback=LightYellow]
  \[f = (-1, 1, 3)\]
  \[g = (3, 1, -1)\]

  \begin{itemize}
          \item Soma: $-1 + 1 + 3 = 3$
          \item Soma dos quadrados: $3^{2} + 1^{2} + (-1)^{2} = 11$
  \end{itemize}
\end{tcolorbox}

\subsubsection*{Questão 3}

\hrulefill

Chamaremos $l$ de lado, $d$ a diagonal e $a$ a área. Montemos a sequência $f = (l, d, a)$. Como o enunciado pede o valor algébrico de $l$, é interessante reescrever todos esses valores em função do mesmo. Utilizando os conhecimentos importados da Geometria Plana, podemos afirmar que a P.A terá a forma

\begin{tcolorbox}[colback=LightYellow]
\[f = (l, l\sqrt{2}, l^{2})\]
\end{tcolorbox}

A P.A se estabelecerá com a existência de $r$. Utilizando a mesma técnica da primeira questão, temos

\begin{tcolorbox}[colback=LightYellow]
\[l\sqrt{2} - l = l^{2} - l\sqrt{2} \]
\[l\sqrt{2} + l\sqrt{2} - l = l^{2}\]
\[2\cdot l\sqrt{2} - l = l^{2}\]
\[l = \frac{2\cdot l\sqrt{2} - l}{l}\]
\[l = \frac{2\cdot l\sqrt{2}}{l} - \frac{l}{l}\]
\[l = 2\sqrt{2} - 1\]
\end{tcolorbox}

\subsubsection*{Questão 4}

\hrulefill

\begin{tcolorbox}[colback=LightYellow]
\begin{itemize}
  \item Hipótese: $\Rightarrow f = (a, b, c)$ é uma P.A
        \vspace{0.2cm}
  \item Tese: $\Rightarrow g = (a^{2}bc, ab^{2}c, abc^{2})$ é uma P.A
\end{itemize}
\end{tcolorbox}

A hipótese afirma que a sequência é, de fato, uma P.A. Isso permite-nos concluir que $b - a = c - b$. Se todos os 3 primeiros elementos forem multiplicados por $abc$, é necessário verificar se essa igualdade mantem-se juntamente com essa proporção. Na P.A da tese, tem-se

\[g = (a^{2}bc, ab^{2}c, abc^{2})\]

Supondo que ela, de fato, é uma P.A, a igualdade $ab^{2}c - a^{2}bc = abc^{2} - ab^{2}c$ é válida. No entanto, isso nada mais é que a igualdade da sequência $f$ multiplicada por $abc$ também. Outro modo de ver isso é considerar o seguinte:

\begin{tcolorbox}[colback=LightYellow]
\[ab^{2}c - a^{2}bc = abc(b - a)\]
\[abc^{2} - ab^{2}c = abc(c - b)\]
\end{tcolorbox}

O que mantem a mesma relação.

\subsubsection*{Questão 5}
\hrulefill

Temos inicialmente $a_{1} = 60$ e uma razão $r$ negativa igual a $-7$. Haverá o enésimo termo a qual $a_{n}$ ficará negativo, que pode ser ilustrado por $a_{n} = 60 + (n - 1)\cdot -7$. Fixaremos $a_{n} < 0$ e trabalharemos com essa inequação.

\[60 + (n - 1)\cdot -7 < 0\]
\[60 - 7n + 7 < 0\]
\[67 - 7n < 0\]
\[-7n < -67\]
\[n > \frac{67}{7} \approx 9.5\]

Como $n$ obrigatoriamente é um número inteiro, admitimos $n \geq 10$. Portanto, a sequência ficará negativa após o décimo termo. Isso é facilmente provado ao verificar $a_{10}$.

\[a_{10} = 60\cdot (9)\cdot -7 = (60 - 63) = (-3)\]

\subsubsection*{Questão 6}
\hrulefill

Podemos definir $a_{10}$ e $a_{12}$ a partir da fórmula do enésimo termo.

\begin{equation*}
\begin{cases}
  a_{10} = a_{1} + 9r \\
  a_{12} = a_{1} + 11r \\
\end{cases}
\end{equation*}

As expressões necessárias para montar a P.A são $a_{1}$ e $r$, montando o sistema e resolvendo-o por soma.

\begin{equation*}
\begin{cases}
  7 = a_{1} + 9r \\
  8 = -a_{1} - 11r \hspace{0.2cm} (\times -1)\\
  \hline
  15 = -2r \\
  r = -\frac{15}{2}
\end{cases}
\end{equation*}

Substituíndo $r$ na primeira equação, encontra-se $a_{1} = \frac{149}{2}$. A formação da P.A pode ser feita com essas duas informações.

\subsubsection*{Questão 7}

\hrulefill

Entre 100 e 1000, sabe-se que há 999 números. Por partes, chamaremos a sequência $f$ de números divisíveis por 2 e $g$ por 3 (não necessariamente o número deve ser divisível por 2 e 3 ao mesmo tempo).

\begin{itemize}
  \item Na sequência $f$, podemos afirmar que $a_{1}$ é igual a 100 (pois é o primeiro número entre 100 e 1000 divisível por 2), $r = 2$ e $a_{n} = 1000$ (último número divisível por 2). É necessário saber quantos $n$ termos essa P.A tem

        \[1000 = 100 + (n - 1)2\]
        \[1000 = 100 + 2n - 2\]
        \[1000 = 98 + 2n\]
        \[2n = 902\]
        \[n = 451\]

  \item A lógica é a mesma para múltiplos de 3. $a_{1} = 102$, $a_{n} = 999$, $r = 3$

        \[999 = 102 + (n - 1)3\]
        \[999 = 102 + 3n - 3\]
        \[999 = 99 + 3n\]
        \[3n = 900\]
        \[n = 300\]
\end{itemize}

\subsubsection*{Questão 8}

\hrulefill

A P.A que representaria a sequência dos 200 primeiros números ímpares positivos é gerada com $a_{1} = 1$ e $r = 2$. Utiliza-se a fórmula

\[S_{n} = \frac{n(a_{1} + a_{n})}{2}\]

Determina-se inicialmente $a_{n} = 1 + (200 - 1)2 \Leftrightarrow 399$, após isso, na fórmula

\[S_{200} = \frac{200(1 + 399)}{2} \Leftrightarrow 100(400) = 40000\]

Para calcularmos a soma de $n$ elementos dessa sequência já estabelecida, seguimos os mesmos passos

\[a_{n} = 1 + 2(n - 1) \Leftrightarrow 2n - 1\]

\[S_{n} = \frac{n(1 + 2n - 1)}{2} \Leftrightarrow \frac{2n^{2}}{2} = n^{2}\]

\subsubsection*{Questão 9}

\hrulefill

A fonte está a 15 metros da primeira parte que será ``regada''. Quando chegar nas primeiras 3 roseiras, andará mais 2 metros para regá-las (vai regar a primeira roseira de imediato, andará até a segunda (1 metro) e até a terceira (1 metro)). Depois disso, terá que voltar para reabastecer a água, voltando todo o caminho novamente, totalizando

\[15 + 2 + 2 + 15 = 34\]

Esse é o trajeto realizado inicialmente, e o primeiro termo da P.A ($a_{1} = 34$). A cada viagem, o percurso aumenta em 6 metros (já que são 3 metros a mais na ida e 3 metros a mais na volta, adicionados com os caminhos anteriores), portanto $r = 6$. Como ele rega 3 roseiras a cada 1 viagem, então ele fará no total 20 viagens para regar todas as 60 ($n = 20$). Montada a P.A, utiliza-se a mesma técnica da questão anterior.

\[a_{20} = 34 + (19)6 = 148\]
\[S_{20} = \frac{20(34 + 148)}{2} = 10(182) = 1820\]

Portanto, o jardineiro andou 1820 metros.

\subsubsection*{Questão 10 (*)}

\hrulefill

A primeira afirmativa da questão diz que o somatório dos 10 primeiros termos da P.A dada é igual a expressão

\[10 + 25d\]

E que o somatório dos 50 primeiros termos dessa mesma P.A equivale à 4550. É possível igualar esses valores com $S_{10}$ e $S_{50}$ para descobrir os termos notáveis.

\[\frac{n(a_{1} + a_{n})}{2} = 10 + 25d\]

Porém $a_{n} = a_{1} + (n - 1)d$ ($d$ é a razão $r$ da questão) e $n = 10$ para esse caso

\[\frac{10(a_{1} + a_{1} + 9d)}{2} = 10 + 25d\]
\[\frac{10(2a_{1} + 9d)}{2} = 10 + 25d\]
\[5(2a_{1} + 9d) = 10 + 25d\]
\[10a_{1} + 45d = 10 + 25d\]
\[10a_{1} = 10 - 20d\]
\[a_{1} = \frac{10 - 20d}{10} \Leftrightarrow 1 - 2d\]

Achado uma expressão que relaciona $a_{1}$ com $d$ para o somatório dos 10 primeiros termos ($a_{1} = 1 - 2d$), agora é necessário descobrir outra relação no somatório de 50 para obter um sistema.

\[\frac{50(a_{1} + a_1 + 49d)}{2} = 4550\]
\[\frac{50(2a_{1} + 49d)}{2} = 4550\]
\[25(2a_{1} + 49d) = 4550\]
\[50a_{1} + 1225d = 4550\]
\[a_{1} = \frac{4550 - 1225d}{50} \Leftrightarrow \frac{4550}{50} - \frac{1225d}{50}\]
\[a_{1} = 91 - \frac{49d}{2}\]

Podemos, por fim, igualar essas duas relações e descobrir o valor de $d$.

\[91 - \frac{49d}{2} = 1 - 2d\]
\[-\frac{49d}{2} = -2d - 90\]
\[\frac{49d}{2} = 2d + 90\]
\[49d = 2(2d + 90)\]
\[49d = 4d + 180\]
\[45d = 180\]
\[d = 4\]

Substituindo $d$ na expressão mais simples $1 - 2d$ teremos $a_{1} = -7$. A expressão pedida é
\[4 - (-7) = 11\]

Letra D.


\section{Progressão Geométrica}

\subsection{Fundamentos}

\subsubsection*{Questão 1}

\hrulefill

\begin{tcolorbox}[colback=LightYellow]
\begin{itemize}
        \item Hipótese $\Rightarrow f = (x, y, z)$ é uma P.G
        \vspace{0.2cm}
        \item Tese $\Rightarrow (x + y + z)(x - y + z) = x^{2} + y^{2} + z^{2}$
\end{itemize}
\end{tcolorbox}

Inicialmente, se $f$ é P.G, então vale a relação

\[\frac{y}{x} = \frac{z}{y}\]

Isso implica $y^{2} = z\cdot x$. Desenvolvendo a tese algebricamente, têm-se

\[(x + y + z)(x - y + z) \Rightarrow (x^{2} - xy + xz) + (yx - y^{2} + yz) + (zx - zy + z^{2})\]

Anulando os termos possíveis, obtemos

\[x^{2} + 2xz - y^{2} + z^{2}\]

No entanto, $y^{2} = zx$.

\[x^{2} + 2y^{2} - y^{2} + z^{2}\]

O que converge para

\[x^{2} - y^{2} + z^{2}\]

\subsubsection*{Questão 2}

\hrulefill

\textbf{Pendente}

\subsubsection*{Questão 3}

\hrulefill


Construindo a P.G tal qual $a_{1} = 2$, $a_{2} = 6$ (é imediato que $q = 3$). Chegaremos ao décimo termo por

\[a_{10} = 2\cdot 3^{9}\]

Ou seja, a expressão dada no enunciado em número é igual a

\[(2\cdot 3^{9})^{\frac{1}{8}} = 3\cdot (2)^{\frac{1}{8}}\]
\[2^{\frac{1}{8}} \cdot 3^{\frac{9}{8}} = 3\cdot (2^{\frac{1}{8}})\]

O que é absurdo, portanto a sentença é falsa.

\subsubsection*{Questão 4}

\hrulefill

A primeira sequência tem como razão:

\[q = \frac{a_{2}}{a_{1}}\]

A segunda sequência têm como razão (considerando que ela é uma P.G)

\[q' = \frac{\frac{1}{a_{2}}}{\frac{1}{a_{1}}} \Rightarrow \frac{a_{1}}{a_{2}}\]

Portanto, ambas são P.G de razão inversa ($q' = q^{-1}$)

\subsubsection*{Questão 5}

\hrulefill

A P.G tem razão $q = 2$ e seu termo $a_{1} = 2^{3}$ (de acordo com a notação de somatório). A determinação do valor de $n$ decorre de que a soma de todos os termos entre $a_{1}$ e $a_{n}$ é igual a 4088, portanto.

\[\frac{a_{1} (q^{n} - 1)}{q - 1} = 4088\]
\[2^{3}(2^{n} - 1) = 4088\]
\[2^{n+3} - 8 = 4088\]
\[2^{n+3} = 4096\]
\[2^{n+3} = 2^{12}\]
\[n = 9\]
