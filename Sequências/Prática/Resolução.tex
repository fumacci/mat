\documentclass[11pt]{article}

\usepackage[svgnames]{xcolor}
\usepackage{tcolorbox}
\usepackage[utf8]{inputenc}
\usepackage{lmodern}
\usepackage{indentfirst}
\usepackage{microtype}
\usepackage{amsmath}
\usepackage[brazil]{babel}
\usepackage{parskip}
\usepackage[tmargin=1.5in, lmargin=1.25in, rmargin=1.25in, bmargin=1.0in]{geometry}

\title{Resolução da L.E (P.A e P.G)}
\author{Mickael Lima}
\date{Dezembro, 2021}

\begin{document}

\maketitle

\section{Progressão Aritmética}

\subsection{Fundamentos}

\subsubsection{Questão 1}

A questão lida com a formação básica de uma P.A. Para determinar o termo $x$ pedido, podemos igualar a constante de razão $r$ da seguinte forma:

\begin{tcolorbox}[colback=LightYellow]
\[(2x + 1) - x = (5x + 7) - (2x + 1)\]
\[x + 1 = 3x + 6\]
\[-2x = 5\]
\[x = -\frac{5}{2}\]
\end{tcolorbox}

A prova real é realizada substituíndo $x$ na sequência e verificando se de fato a mesma forma uma P.A com os três primeiros termos dados.

\begin{tcolorbox}[colback=LightYellow]
\[f = \left(-\frac{5}{2}, -2\cdot \frac{5}{2} + 1, -5\cdot \frac{5}{2} + 7 \right)\]
\[f = \left(-2.5, -4, -5.5\right)\]
\end{tcolorbox}

De fato, forma-se a P.A de $a_{1} = (-2.5)$ e $r = (-1.5)$.

\subsubsection{Questão 2}

O enunciado pede duas condições a ser cumpridas (além da formação em P.A)

\begin{itemize}
  \item A soma dos 3 números deve ser igual a 3
  \item A soma dos 3 números, cada um ao quadrado, deve ser igual a 11
\end{itemize}

Isso implica no seguinte sistema

\begin{tcolorbox}[colback=LightYellow]
\begin{equation*}
\begin{cases}
  (x - r) + x + (x + r) = 3 \\
  (x - r)^2 + x^2 + (x + r)^2 = 11 \\
\end{cases}
\end{equation*}
\end{tcolorbox}

Do primeiro membro, nota-se inicialmente que $r$ irá ser cancelado pela soma entre $-r$ e $+r$. Ficando apenas $3x = 3$. Desse modo, descobre-se $x = 1$. Jogando essa informação no segundo membro, poderemos manipular e descobrir $r$ efetivamente.

\begin{tcolorbox}[colback=LightYellow]
\[(1 - r)^{2} + 1 + (1 + r)^{2} = 11\]
\[(1^{2} - 2r + r^{2}) + 1 + (1^{2} + 2r + r^{2}) = 11\]
\end{tcolorbox}

\begin{tcolorbox}[colback=LightYellow]
\[3 + 2r^{2} = 11 \]
\[r = \pm 2\]
\end{tcolorbox}

Sendo $x = 1$ e $r = \pm 2$, podemos montar duas progressões aritméticas $f$ (para $r = 2$) e $g$ (para $r = -2$), como ilustrado abaixo.


\begin{tcolorbox}[colback=LightYellow]
  \[f = (-1, 1, 3)\]
  \[g = (3, 1, -1)\]

  \begin{itemize}
          \item Soma: $-1 + 1 + 3 = 3$
          \item Soma dos quadrados: $3^{2} + 1^{2} + (-1)^{2} = 11$
  \end{itemize}
\end{tcolorbox}

\subsubsection{Questão 3}

Chamaremos $l$ de lado, $d$ a diagonal e $a$ a área. Montemos a sequência $f = (l, d, a)$. Como o enunciado pede o valor algébrico de $l$, é interessante reescrever todos esses valores em função do mesmo. Utilizando os conhecimentos importados da Geometria Plana, podemos afirmar que a P.A terá a forma

\begin{tcolorbox}[colback=LightYellow]
\[f = (l, l\sqrt{2}, l^{2})\]
\end{tcolorbox}

A P.A se estabelecerá com a existência de $r$. Utilizando a mesma técnica da primeira questão, temos

\begin{tcolorbox}[colback=LightYellow]
\[l\sqrt{2} - l = l^{2} - l\sqrt{2} \]
\[l\sqrt{2} + l\sqrt{2} - l = l^{2}\]
\[2\cdot l\sqrt{2} - l = l^{2}\]
\[l = \frac{2\cdot l\sqrt{2} - l}{l}\]
\[l = \frac{2\cdot l\sqrt{2}}{l} - \frac{l}{l}\]
\[l = 2\sqrt{2} - 1\]
\end{tcolorbox}

\subsubsection{Questão 4}

\begin{tcolorbox}[colback=LightYellow]
\begin{itemize}
  \item Hipótese: $\Rightarrow f = (a, b, c)$ é uma P.A
  \item Tese: $\Rightarrow g = (a^{2}bc, ab^{2}c, abc^{2})$ é uma P.A
\end{itemize}
\end{tcolorbox}

A hipótese afirma que a sequência é, de fato, uma P.A. Isso permite-nos concluir que $b - a = c - b$. Se todos os 3 primeiros elementos forem multiplicados por $abc$, é necessário verificar se essa igualdade mantem-se juntamente com essa proporção. Na P.A da tese, tem-se

\[g = (a^{2}bc, ab^{2}c, abc^{2})\]

Supondo que ela, de fato, é uma P.A, a igualdade $ab^{2}c - a^{2}bc = abc^{2} - ab^{2}c$ é válida. No entanto, isso nada mais é que a igualdade da sequência $f$ multiplicada por $abc$ também. Outro modo de ver isso é considerar o seguinte:

\begin{tcolorbox}[colback=LightYellow]
\[ab^{2}c - a^{2}bc = abc(b - a)\]
\[abc^{2} - ab^{2}c = abc(c - b)\]
\end{tcolorbox}

O que mantem a mesma relação.

\subsubsection{Questão 5}

Temos inicialmente $a_{1} = 60$ e uma razão $r$ negativa igual a $-7$. Haverá o enésimo termo a qual $a_{n}$ ficará negativo, que pode ser ilustrado por $a_{n} = 60 + (n - 1)\cdot -7$. Fixaremos $a_{n} < 0$ e trabalharemos com essa inequação.

\[60 + (n - 1)\cdot -7 < 0\]
\[60 - 7n + 7 < 0\]
\[67 - 7n < 0\]
\[-7n < -67\]
\[n > \frac{67}{7} \approx 9.5\]

Como $n$ obrigatoriamente é um número inteiro, admitimos $n \geq 10$. Portanto, a sequência ficará negativa após o décimo termo. Isso é facilmente provado ao verificar $a_{10}$.

\[a_{10} = 60\cdot (9)\cdot -7 = (60 - 63) = (-3)\]

\subsubsection{Questão 6}

Podemos definir $a_{10}$ e $a_{12}$ a partir da fórmula do enésimo termo.

\begin{equation*}
\begin{cases}
  a_{10} = a_{1} + 9r \\
  a_{12} = a_{1} + 11r \\
\end{cases}
\end{equation*}

As expressões necessárias para montar a P.A são $a_{1}$ e $r$, montando o sistema e resolvendo-o por soma.

\begin{equation*}
\begin{cases}
  7 = a_{1} + 9r \\
  8 = -a_{1} - 11r \hspace{0.2cm} (\times -1)\\
  \hline
  15 = -2r \\
  r = -\frac{15}{2}
\end{cases}
\end{equation*}

Substituíndo $r$ na primeira equação, encontra-se $a_{1} = \frac{149}{2}$. A formação da P.A pode ser feita com essas duas informações.

\subsubsection{Questão 7}

Entre 100 e 1000, sabe-se que há 999 números. Por partes, chamaremos a sequência $f$ de números divisíveis por 2 e $g$ por 3 (não necessariamente o número deve ser divisível por 2 e 3 ao mesmo tempo).

\begin{itemize}
  \item Na sequência $f$, podemos afirmar que $a_{1}$ é igual a 100 (pois é o primeiro número entre 100 e 1000 divisível por 2), $r = 2$ e $a_{n} = 998$ (último número divisível por 2). É necessário saber quantos $n$ termos essa P.A tem

        \[1000 = 100 + (n - 1)2\]
        \[1000 = 100 + 2n - 2\]
        \[1000 = 98 + 2n\]
        \[2n = 902\]
        \[n = 451\]

  \item A lógica é a mesma para múltiplos de 3. $a_{1} = 102$, $a_{n} = 999$, $r = 3$

        \[999 = 102 + (n - 1)3\]
        \[999 = 102 + 3n - 3\]
        \[999 = 99 + 3n\]
        \[3n = 900\]
        \[n = 300\]
\end{itemize}




\end{document}
