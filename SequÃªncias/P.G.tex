\documentclass[11pt]{article}

\usepackage[svgnames]{xcolor}
\usepackage{tcolorbox}
\usepackage[utf8]{inputenc}
\usepackage{lmodern}
\usepackage{indentfirst}
\usepackage{microtype}
\usepackage{amsmath}
\usepackage[brazil]{babel}
\usepackage{parskip}
\usepackage[tmargin=1.5in, lmargin=1.25in, rmargin=1.25in, bmargin=1.0in]{geometry}

\title{Sequências e Progressões II}
\author{Mickael Lima}
\date{Dezembro, 2021}

\begin{document}

\maketitle
\pagebreak
\tableofcontents
\pagebreak

\section{Progressão Geométrica}

A progressão geométrica é outro tipo de progressão, semelhante à aritmética. É definida pela fórmula de recorrência ilustrada por

\begin{equation*}
\begin{cases}
  a_{1} \\
  a_{n} = a_{n - 1} \cdot q
\end{cases}
\end{equation*}

Sendo $a, q$ números reais fornecidos (ou pelo menos implicito, no caso de $q$). Sendo assim, uma P.G é toda sequência a qual o próximo termo $a_{n}$ seja igual ao termo anterior $a_{n - 1}$ multiplicado por uma constante $q$ (que equivale ao $r$ da P.A).

\subsection{Classificações}

As P.Gs são classificadas de 5 modos diferentes.

\begin{itemize}
        \item Crescente: o próximo termo é maior que o anterior
        \begin{equation*}
        \begin{align*}
          a_{n} &> a_{n - 1} \\
          a_{n - 1} \cdot q &> a_{n - 1} \\
          q &> \frac{a_{n - 1}}{a_{n - 1}} \\
          q &> 1 \\
        \end{align*}
        \end{equation*}
\end{itemize}

Nesse caso, a P.G será crescente quando $q > 1$ \textbf{e somente para sequências positivas}. Para casos em que $a_{1}$ é negativo, vale a relação $0 < q < 1$.
\begin{itemize}
  \item Constante: Há duas situações em que isso acontece.
        \begin{itemize}
                \item Quando $q = 1$ (já que não haverá variação na multiplicação de $a_{n}$ e $a_{n - 1}\cdot q$).
                \item Quando $a_{1} = 0$ (já que multiplicar por zero a deixará constante).
        \end{itemize}
\end{itemize}

\begin{itemize}
  \item Decrescente: há dois casos para se analisar (semelhante à crescente).
        \begin{itemize}
                \item P.G positiva: será decrescente para $0 < q < 1$
                \item P.G negativa: será decrescente para $q > 1$
        \end{itemize}
\end{itemize}

\begin{itemize}
  \item Alternantes: o próximo termo tem sinal contrário ao anterior. Isso ocorre sempre que $q < 0$, forçando a alternância de sinais.

  \item Estacionárias: Quando $q = 0$ e $a_{1} \neq 0$, forçando-a a ficar constante após o primeiro termo.
\end{itemize}



\end{document}
