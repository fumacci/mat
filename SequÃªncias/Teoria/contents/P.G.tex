\section{Progressão Geométrica}

A progressão geométrica é outro tipo de progressão, semelhante à aritmética. É definida pela fórmula de recorrência ilustrada por

\begin{equation*}
\begin{cases}
  a_{1} \\
  a_{n} = a_{n - 1} \cdot q
\end{cases}
\end{equation*}

Sendo $a, q$ números reais fornecidos (ou pelo menos implicito, no caso de $q$). Sendo assim, uma P.G é toda sequência a qual o próximo termo $a_{n}$ seja igual ao termo anterior $a_{n - 1}$ multiplicado por uma constante $q$ (que equivale ao $r$ da P.A).

\subsection{Classificação}

As P.Gs são classificadas de 5 modos diferentes.

\begin{tcolorbox}[colback=LightYellow]

\begin{itemize}
        \item Crescente: o próximo termo é maior que o anterior


      \begin{align*}
        a_{n} &> a_{n - 1} \\
        a_{n - 1} \cdot q &> a_{n - 1} \\
        q &> \frac{a_{n - 1}}{a_{n - 1}} \\
        q &> 1
      \end{align*}
\end{itemize}

Nesse caso, a P.G será crescente quando $q > 1$ \textbf{e somente para sequências positivas}. Para casos em que $a_{1}$ é negativo, vale a relação $0 < q < 1$.
\end{tcolorbox}

\begin{tcolorbox}[colback=LightYellow]
\begin{itemize}
\item Constante: Há duas situações em que isso acontece.
      \begin{itemize}
              \item Quando $q = 1$ (já que não haverá variação na multiplicação de $a_{n}$ e $a_{n - 1}\cdot q$).
              \item Quando $a_{1} = 0$ (já que multiplicar por zero a deixará constante).
      \end{itemize}
\end{itemize}
\end{tcolorbox}

\begin{tcolorbox}[colback=LightYellow]
\begin{itemize}
\item Decrescente: há dois casos para se analisar (semelhante à crescente).
      \begin{itemize}
              \item P.G positiva: será decrescente para $0 < q < 1$
              \item P.G negativa: será decrescente para $q > 1$
      \end{itemize}
\end{itemize}
\end{tcolorbox}

\begin{tcolorbox}[colback=LightYellow]
\begin{itemize}
\item Alternantes: o próximo termo tem sinal contrário ao anterior. Isso ocorre sempre que $q < 0$, forçando a alternância de sinais.

\item Estacionárias: Quando $q = 0$ e $a_{1} \neq 0$, forçando-a a ficar constante após o primeiro termo.
\end{itemize}
\end{tcolorbox}

\subsection{Notações Úteis}

Tal qual descrito nas notas sobre P.A, as notações úteis da P.G podem ser escritas (como exemplo, os 3 primeiros termos da P.G) como.

\[\left(x, x\cdot q, x\cdot q^{2}\right)\]
\[\left(\frac{x}{q}, x, x\cdot q\right)\]

Para 4 termos, têm-se $\left(x, xq, xq^{2}, xq^{3}\right)$, para $n$ termos, $\left(x, xq, xq^{2}, \dots, xq^{n - 1}\right)$

\begin{tcolorbox}[colback=LightYellow]
\begin{itemize}
  \item Qual número deverá ser somado a 1, 9 e 15 para termos, nessa ordem, três números em P.G.
\end{itemize}
\end{tcolorbox}

\begin{tcolorbox}[colback=LightYellow]
A sequência em P.G $f$ terá a forma de $f = \{(1 + x), (9 + x), (15 + x)\}$. O próximo termo deverá ser igual ao produto entre o termo anterior com uma constante $q$. Para que essa constante $q$ exista, é estabelecida a seguinte relação.

\[q = \frac{9 + x}{1 + x} = \frac{15 + x}{9 + x}\]

Portanto, constrói-se a seguinte expressão

\end{tcolorbox}
\begin{tcolorbox}[colback=LightYellow]
\[(9 + x)^{2} = (15 + x)(1 + x)\]
\[81 + 18x + x^{2} = 15 + 15x + x + x^{2}\]
\[81 + 18x = 15 + 16x\]
\[2x = -66\]
\[x = -33\]
\end{tcolorbox}

\subsection{Fórmula do enésimo termo}

Semelhante à progressão aritmética, a P.G pode ser armada do termo $a_{1}$ até $a_{n}$ ($a_{1} \neq 0$, $q \neq 0$, $n$ conhecido) da seguinte forma.

\begin{equation*}
\begin{cases}
a_{2} = a_{1}\cdot q \\
a_{3} = a_{2}\cdot q \\
\dots \\
a_{n} = a_{n - 1} \cdot q
\end{cases}
\end{equation*}

Caso o primeiro lado ($a_{2}, \dots$) seja multiplicado, e o segundo lado $(a_{1}\cdot q), (a_{2}\cdot q), \dots$ também, é evidente que a igualdade se manterá, formando a seguinte equação.

\[a_{2} \cdot a_{3}\dots a_{n} = (a_1\cdot q) \cdot (a_2\cdot q) \cdot (a_3\cdot q) \dots (a_{n - 1} \cdot q)\]

É possível pôr o $q$ em evidência, visto que ele aparece $n - 1$ vezes ($a_{n}$ não é contado) no segundo membro.

\[[a_{2} \cdot a_{3}\dots, a_{n - 1}]\cdot a_{n} = (a_1 [\cdot a_2 \cdot a_3 \dots a_{n - 1}])\cdot q^{n - 1}\]

O que está destacado por colchetes se cancelam, formando a equação final em função de $n$.


\begin{tcolorbox}[colback=LightYellow]
\[a_{n} = a_1\cdot q^{n - 1}\]
\end{tcolorbox}

\subsubsection{Demonstração por Indução Finita}

\begin{itemize}
\begin{tcolorbox}[colback=LightYellow]
\item Checa-se a validade para $n = 1$

\[a_{1} = a_{1}\cdot q^{1-1}\]
\[a_{1} = a_{1}\]
\end{tcolorbox}

\begin{tcolorbox}[colback=LightYellow]
\item Admite-se válido para $n = p$
      \[a_{p} = a_{1}\cdot q^{p - 1}\]
\end{tcolorbox}

\begin{tcolorbox}[colback=LightYellow]
\item Checa-se a validade para $n = p + 1$
      \[a_{p + 1} = a_{1}\cdot q^{(p + 1) - 1} \Leftrightarrow a_{1}\cdot q^{p-1}\cdot q\]
\end{tcolorbox}
\end{itemize}

\subsection{Interpolação Geométrica}

A mesma definição dada para a interpolação aritmética vale para a interpolação geométrica. A razão $q$ deverá ser isolada e os extremos $a_{1}$ e $a_{n}$ devem ser definidos

\begin{tcolorbox}[colback=LightYellow]
\[a_{n} = a_{1}\cdot q^{n - 1}\]
\[q^{n - 1} = \frac{a_{n}}{a_{1}}\]
\[q = \sqrt[(n - 1)]{\frac{a_{n}}{a_{1}}}\]
\end{tcolorbox}

Porém, consideram-se os extremos, então o termo $n - 1$ deverá ser reescrito como $n - 1 + 2 = n + 1$

\begin{tcolorbox}[colback=LightYellow]
\[q = \sqrt[(n + 1)]{\frac{a_{n}}{a_{1}}}\]
\end{tcolorbox}

\begin{tcolorbox}[colback=LightYellow]
\begin{itemize}
        \item Interpolar 8 meios geométricos entre 5 e 2560
\end{itemize}
\end{tcolorbox}


\begin{tcolorbox}[colback=LightYellow]
  Inicialmente, define-se $a_{1} = 5$ e $a_{10} = 2560$. A razão $q$ é definida por

  \[q = \sqrt[8 + 1]{\frac{2560}{5}} = \sqrt[9]{512} = 2\]

  Gera-se a P.G $f$

  \[f = (5 , 10, 20, 40, 80, 160, 320, 640, 1280, 2560)\]
\end{tcolorbox}

\subsection{Produto de uma P.G de $n$ termos}

Inicialmente, é importante relembrar da fórmula da soma de $n$ números inteiros (detalhada na parte de P.A)

\begin{tcolorbox}[colback=LightYellow]
\[S_{n} = \frac{n(n + 1)}{2}\]
\end{tcolorbox}

A partir disso, define-se que o produto de uma P.G de $n$ termos é igual a:

\begin{tcolorbox}[colback=LightYellow]
\[P_{n} = (a_{1})^{n} \cdot q^{\frac{n(n + 1)}{2}}\]
\end{tcolorbox}

Isso acontece pela relação sistemática de multiplicação abaixo (detalhada na parte de P.A, porém na soma).

\begin{tcolorbox}[colback=LightYellow]
\begin{equation*}
\begin{cases}
  a_{1} = a_{1} \\
  a_{2} = a_{1}\cdot q \\
  a_{3} = a_{1}\cdot q^{2} \\
  \dots \\
  a_{n} = a_{1} \cdot q^{n - 1} \\
  \hline
  a_{1}\cdot a_{2}\cdot a_{3}\cdot \dots \cdot a_{n} = (a_{1}\cdot a_{1}\cdot \dots \cdot a_{1})(q^{1}\cdot q^{2}\cdot \dots q^{n - 1})
\end{cases}
\end{equation*}
\end{tcolorbox}

O produto entre $q$ pode ser resumida usando propriedades de expoentes. O lado esquerdo é o ``produto entre termos'' isolado. A equação ficará.

\[(a_{1})^{n}\cdot q^{(1 + 2 + 3 + \dots + (n - 1))}\]

Como o expoente de $q$ é nada mais que a soma de $n - 1$ inteiros, voltaremos a fórmula inicial.

\subsection{Soma dos $n$ termos da P.G}

Inicialmente, a soma de uma P.G finita de $n$ termos é dada por

\[S_{n} = (a_{1}) + (a_{1}\cdot q) + (a_{1}\cdot q^{2}) + \dots + (a_{1}\cdot q^{n - 2}) + (a_{1}\cdot q^{n - 1})\]

Se por conveniência, ambos os lados dessa equação for multiplicada por $q$

\[q\cdot S_{n} = (a_{1}\cdot q) + (a_{1}\cdot q^{2}) + \dots + (a_{1}\cdot q^{n - 1}) + (a_{1}\cdot q^{n})\]

Observado que $a_{1}$ (isolado) aparece apenas na primeira equação, e $a_{1}\cdot q^{n}$ aparece apenas na segunda, é possível ``somar'' como se fosse um sistema para eliminar o restante em comum. Já que a segunda opção é maior que a primeira $q\cdot S_{n} > S_{n}$ é preferível subtrair na ordem (2) - (1).

\begin{tcolorbox}[colback=LightYellow]
\begin{equation*}
  \begin{cases}
    q\cdot S_{n} = (a_{1}\cdot q) + (a_{1}\cdot q^{2}) + \dots + (a_{1}\cdot q^{n - 1}) + (a_{1}\cdot q^{n}) \\
    S_{n} = (a_{1}) + (a_{1}\cdot q) + (a_{1}\cdot q^{2}) + \dots + (a_{1}\cdot q^{n - 2}) + (a_{1}\cdot q^{n - 1}) \\
  \end{cases}
\end{equation*}

\[q\cdot S_{n} - S_{n} = a_{1}\cdot q^{n} - a_{1}\]

Colocando $S_{n}$ em evidência no primeiro membro

\[S_{n} (q - 1) = a_{1}\cdot q^{n} - a_{1}\]

\end{tcolorbox}
Considerando $q \neq 1$, chegaremos na seguinte expressão

\[S_{n} = \frac{a_{1} \cdot q^{n} - a_{1}}{(q - 1)}\]

\section{Progressão Geométrica Infinita}

\subsection{Limite de uma Sequência}

Considerando uma progressão geométrica com $a_{1} = \frac{1}{2}$ e $q = \frac{1}{2}$ com extremo até o infinito. Os 4 primeiros termos são dados por

\begin{tcolorbox}[colback=LightYellow]
\[f = \left(\frac{1}{2}, \frac{1}{4}, \frac{1}{8}, \frac{1}{16}, \dots, \frac{1}{2^{n}}, \dots \right)\]
\end{tcolorbox}

É perceptível que $n$ e $a_{n}$ são valores inversamente proporcionais. Ou seja, quanto maior $n$, menor será $a_{n}$. A partir de um certo ponto, $n$ será tão grande a ponto de $a_{n}$ chegar muito perto de 0. Um exemplo disso: podemos computar a partir de qual termo $n$ os valores de $a_{n}$ serão menores que $0.001$

\begin{tcolorbox}[colback=LightYellow]
\[\frac{1}{2^{n}} - 0 < \frac{1}{1000}\]
\[2^{n} > 1000\]
\end{tcolorbox}

Nesse contexto, determinamos que o elemento da sequência $a_{n}$ ficará menor que 0.001 em algum $n$ entre 9 e 10. Porém, como $n$ é obrigatorialmente inteiro, teremos certeza que ficará menor a partir de $n > 9$ (já que $2^{9} = 512$). Se chamarmos a aproximação pedida (no caso, foi 0.001) de $\varepsilon$ ($> 0$), é possível encontrar um número $n_{o}$ natural tal que

\[\frac{1}{2^{n}} - 0 < \varepsilon\]

Quando $n > n_{o}$. Sendo $n$ o número real (que no caso anterior seria $n$ tal qual $9 < n < 10$) e $n_{o}$ o número natural usado no final (no caso anterior $n = 9$). A sequência dada se aproximará de 0 quando $n$ tender ao infinito.

\begin{tcolorbox}[colback=LightYellow]
\[\lim_{n\to \infty} \frac{1}{2^{n}} = 0\]
\end{tcolorbox}

\subsubsection{Definição}

Uma P.G $f = (a_{1}, a_{2}, a_{3}, \dots, a_{n}, \dots)$ tem um limite $l$ se dado $\varepsilon > 0$, é possível obter um número $n_{o}$ natural tal que $|a_{n} - l| < \varepsilon$ quando $n > n_{o}$. Caso exista, afirma-se que a sequência $f$ converge-se para $l$.

\begin{tcolorbox}[colback=LightYellow]
Toda a sequência $f = (1, q^{2}, q^{3}, \dots, q^{n}, \infty)$ com $-1 < q < 1$, o elemento $a_{n}$ com o aumento de $n$ sempre tenderá a 0.
\end{tcolorbox}

\subsection{Soma de termos da P.G infinita}

\subsubsection{Exemplo preliminar}

Considere a P.G infinita

\begin{tcolorbox}[colback=LightYellow]
\[f = \left(\frac{1}{2}, \frac{1}{4}, \frac{1}{8}, \frac{1}{16}, \dots, \frac{1}{2^{n}}, \dots \right)\]
\end{tcolorbox}

Formemos uma sequência com a soma de $a_{1} + a_{2} + \dots + a_{n}$ termos.

\[g = (S_{1}, S_{2}, \dots)\]

Sendo

\begin{equation*}
  \begin{cases}
    S_{1} = \frac{1}{2} \\
    S_{2} = \frac{1}{2} + \frac{1}{4} = \frac{3}{4} \\
    S_{3} = \frac{1}{2} + \frac{1}{4} + \frac{1}{8} = \frac{7}{8} \\
    \dots
  \end{cases}
\end{equation*}

A expressão $S_{n}$ será dada pela soma de tudo até esse elemento, simbolizada nessa sequência por

\[S_{n} = \frac{1}{2} + \frac{1}{4} + \frac{1}{8} + \dots + \frac{1}{2^{n}}\]

Observando-se a sequência de somas anteriores, é visto um padrão entre o número do denominador e o do numerador. No exemplo $S_{2}$, a soma resultou em $\frac{3}{4}$, em $S_{3} = \frac{7}{8}$, ou seja, é estabelecido algo como

\[S_{n} = \frac{n - 1}{n}\]

No entanto, esse $n$ nada mais é que $2^{n}$ (notando que o denominador sempre se conserva após a soma), portanto ficaremos com

\[S_{n} = \frac{2^{n} - 1}{2^{n}} \Leftrightarrow 1 - \frac{1}{2^{n}}\]

De fato, podemos definir para qual valor $l$ a sequência de somas tenderá (e por consequência, o valor da soma desses $n$ termos) quanto maior $n$

\[\lim_{n \to +\infty} S_{n} \Leftrightarrow \lim_{n\to +\infty} \left(1 - \frac{1}{2^{n}}\right)\]
\[\lim_{n\to +\infty} (1) - \lim_{n\to +\infty} \left(\frac{1}{2^{n}}\right) \Leftrightarrow l = 1\]

\subsubsection{Teorema da Soma}

Em resumo, seja uma P.A infinita genérica (obecedendo os termos já colocados), a soma dos termos (o limite $l$ da sequência de soma) será sempre igual a expressão

\begin{tcolorbox}[colback=LightYellow]
\[S_{n} = \frac{a_{1}}{1 - q}\]
\end{tcolorbox}
