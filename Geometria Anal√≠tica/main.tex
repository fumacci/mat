\documentclass[11pt]{memoir}

% -- color boxes
\usepackage[svgnames]{xcolor}
\usepackage{tcolorbox}

% -- formatting stuff
\usepackage[tmargin=1.5in, lmargin=1.25in, rmargin=1.25in, bmargin=1.0in]{geometry}
\usepackage[utf8]{inputenc}
\usepackage{parskip}

% -- math stuff
\usepackage{amsmath, amsfonts, amssymb}
\usepackage{tikz}

% -- language and font
\usepackage{mathptmx}
\usepackage{microtype}
\usepackage{indentfirst}
\usepackage[brazil]{babel}

% -- cool chapter
\makechapterstyle{box}{
  \renewcommand*{\printchaptername}{}
  \renewcommand*{\chapnumfont}{\normalfont\sffamily\huge\bfseries}
  \renewcommand*{\printchapternum}{
    \flushright
    \begin{tikzpicture}
      \draw[fill,color=black] (0,0) rectangle (2cm,2cm);
      \draw[color=white] (1cm,1cm) node { \chapnumfont\thechapter };
    \end{tikzpicture}
  }
  \renewcommand*{\chaptitlefont}{\normalfont\sffamily\Huge\bfseries}
  \renewcommand*{\printchaptertitle}[1]{\flushright\chaptitlefont##1}
}

\chapterstyle{box}

\begin{document}

\chapter{Coordenadas}

As coordenadas de um ponto P são descritas por dois números reais $x, y$ que indicam a posição em relação ao eixo $x$ e $y$ no sistema cartesiano (esse modelo pode ser chamado de ortogonal, ortonormal ou retangular). Dizer que um ponto P localiza-se em $P(3, 4)$ significa que ele está em $x = 3$ (reta do eixo $x$) e $y = 4$ (reta do eixo $y$). Um adendo importante: \textbf{todo ponto P representa um conjunto de duas icógnitas a serem achadas}.

\section{Relação entre P e $(p_{x}, p_{y})$}


\begin{tcolorbox}[colback=LightYellow]
  Entre o conjunto de pontos P do plano cartesiano e o conjunto de pares ordenados $(p_{x}, p_{y}) \in \mathbb{R}$, existe uma correspondência biunívoca (ou seja, para $P$ existe apenas, e somente apenas um $(p_{x}, p_{y})$ e vice-versa).
\end{tcolorbox}

Isso implica que as coordenadas $(a, b)$ são diferentes das $(b, a)$, ou seja, a ordem de $a, b$ importa\footnote{A demonstração dessa afirmação, e por conseguinte dessa propriedade, é feita no FME 7, página 10}.

\section{Distância entre dois pontos}

\begin{tcolorbox}[colback=LightYellow]
  Dado dois pontos $P_{1}$ e $P_{2}$, com coordenadas $(x_{1}, y_{1})$ e $(x_{2}, y_{2})$ respectivamente, a distância entre esses pontos é dada pela seguinte equação.

  \begin{equation} \label{eq:d}
    d = \sqrt{(\Delta x)^{2} + (\Delta y)^{2}}
  \end{equation}

  Onde $\Delta x$ e $\Delta y$ é a variação $x_{2} - x_{1}$, $y_{2} - y_{1}$
\end{tcolorbox}

\subsection{Casos e Demonstração da Fórmula}

Há dois casos em que o cálculo da distância entre os pontos é intuitiva.

\begin{itemize}
        \item O segmento de reta formado é paralelo ao eixo $x$. Nesse caso, $d = |x_{2} - x_{1}|$
\end{itemize}

\begin{center}
  \begin{tikzpicture}
    \draw[thick, ->] (-1, 0) -- (4, 0);
    \draw[thick, ->] (0, -1) -- (0, 3);
    \draw[thick] (1, 1.5) -- (2.5, 1.5);
    \draw[dashed] (1, 1.5) -- (1, 0) node[anchor=north] {$x_{1}$};
    \draw[dashed] (2.5, 1.5) -- (2.5, 0) node[anchor=north] {$x_{2}$};
    \filldraw[black] (1, 1.5) circle (2pt);
    \filldraw[black] (2.5, 1.5) circle (2pt);
  \end{tikzpicture}
\end{center}

\begin{itemize}
  \item O segmento de reta é paralelo ao eixo $y$. Nesse caso, $d = |y_{2} - y_{1}|$.
\end{itemize}

\begin{center}
  \begin{tikzpicture}
    \draw[thick, ->] (-1, 0) -- (4, 0);
    \draw[thick, ->] (0, -1) -- (0, 3);
    \draw[thick] (1.5, 1) -- (1.5, 2.5);
    \draw[dashed] (1.5, 1) -- (0, 1) node[anchor=east] {$y_{1}$};
    \draw[dashed] (1.5, 2.5) -- (0, 2.5) node[anchor=east] {$y_{2}$};
    \filldraw[black] (1.5, 1) circle (2pt);
    \filldraw[black] (1.5, 2.5) circle (2pt);
  \end{tikzpicture}
\end{center}

A demonstração da fórmula decorre desses dois casos. Considerando que os pontos não estejam paralelos ao eixo e o segmento de reta esteja inclinado, formando um triângulo retângulo, como representado abaixo.

\begin{center}
  \begin{tikzpicture}
    \draw[thick, ->] (-1, 0) -- (4, 0);
    \draw[thick, ->] (0, -1) -- (0, 3);
    \draw[thick] (0.5, 0.5) -- (2.4, 2.4);
    \draw[dashed] (0.5, 0.5) -- (2.4, 0.5);
    \draw[dashed] (2.4, 2.4) -- (2.4, 0.5);
    \filldraw[black] (0.5, 0.5) circle (2pt) node[anchor=east] {A};
    \filldraw[black] (2.4, 2.4) circle (2pt) node[anchor=west] {B};
    \filldraw[gray] (2.4, 0.5) circle (1.5pt) node[anchor=west] {C};
  \end{tikzpicture}
\end{center}

Surge um novo ponto $C$ no ângulo reto do triângulo formado. $\overline{AC}$ e $\overline{BC}$ são paralelos aos eixos, e as coordenadas de $C$ são $(x_{B}, y_{A})$. Os casos demonstrados anteriormente dão a expressão que indica a medida de $\overline{AC}$ e $\overline{BC}$. Aplicando o Teorema de Pitágoras com essas informações chega-se na fórmula \ref{eq:d} mostrada no começo da sessão.

\subsubsection{Questões Exemplo}

\begin{tcolorbox}[colback=LightYellow]
1. (\textbf{UFR-RJ}) A palavra ``perímetro'' vem da combinação de dois elementos gregos: o primeiro \emph{peri}, significa ``em torno de'', e o segundo, \emph{metron}, significa medida. O perímetro do trapézio cujo vértices têm coordenadas $(-1, 0), (9, 0), (8, 5)$ e $(1,5)$ é igual a:
\end{tcolorbox}

A questão pede a soma dos lados de um trapézio. De fato, é só computar a medida $d$ dos 4 lados em função de seus pontos.

\begin{align*}
  d_{1} &= \sqrt{(9 - [-1])^{2} + (0)^{2}} \Rightarrow \sqrt{100} = 10 \\
  d_{2} &= \sqrt{(9 - 8)^{2} + 5^{2}} \Rightarrow \sqrt{26} \\
  d_{3} &= \sqrt{(8 - 1)^{2} + 0^{2}} \Rightarrow \sqrt{49} = 7 \\
  d_{4} &= \sqrt{(1 - [-1])^{2} + 5^{2}} \Rightarrow \sqrt{29}
\end{align*}

Notando que, em $d_{4}$ trabalha-se com os pontos $(1, 5)$ e $(-1, 0)$, já que é necessário ``fechar'' esse trapézio. Portanto o perímetro será igual $10 + \sqrt{26} + 7 + \sqrt{29} \Rightarrow 17 + \sqrt{26} + \sqrt{29}$.

\begin{tcolorbox}[colback=LightYellow]
2. Dados A$(1, 2)$, C$(3, -4)$ extremidades da diagonal de um quadrado, determine as coordenadas do vértice $B$ e $D$, sabendo que $x_{B} > x_{D}$. \footnote{Essa questão foi retirada diretamente da Universidade Federal Clube de Regatas do Flamengo e elaborada pelo Jorge Jesus em parceria com o Bruno Henrique e Gabigol}
\end{tcolorbox}

Sendo $\overline{AC}$ a diagonal, é esperado que os lados desse quadrado sejam iguais a $d_{AC}\cdot \sqrt{2}$, e portanto a distância entre todos os vértices do mesmo. Computando $d_{AC}$:

\begin{equation}
  d_{AC} = \sqrt{(3 - 1)^{2} + (-4 - 2)^{2}} \Rightarrow \sqrt{4 + 36} = \sqrt{40}
\end{equation}

Portanto os lados terão medida $\sqrt{80}$. Dessa informação deriva-se que $d_{AB} = d_{BC} = d_{CD} = \sqrt{80}$. Escrevendo as expressões para determinar as coordenadas de $B$.

\[\sqrt{(x_{b} - x_{a})^{2} + (y_{b} - y_{a})^{2}}\]
\[(x_{b} - x_{a})^{2} + (y_{b} - y_{a})^{2}\]
\[(x_{b} - 1)^{2} + (y_{b} - 2)^{2}\]

A expressão acima, apesar de conter duas icógnitas, será resolvível quando montada a expressão para as coordenadas $DS$. Efetuando os produtos notáveis.

\[(x_{b}^{2} - 2x_{b} + 1) + (y_{b}^{2} - 4y_{b} + 4)\]
\[x_{b}^{2} - 2x_{b}  + y_{b}^{2} - 4y_{b} + 5\]
\[x_{b}(x_{b} - 2) + y_{b}(y_{b} - 4) + 5\]



\end{document}
