\documentclass[12pt]{article}

\usepackage[utf8]{inputenc}
\usepackage{lmodern}
\usepackage{indentfirst}
\usepackage{microtype}
\usepackage{amsmath}
\usepackage[brazil]{babel}
\usepackage[tmargin=1.5in, lmargin=1.25in, rmargin=1.25in, bmargin=1.0in]{geometry}
\title{Sequências e Progressões}
\author{Mickael Lima}
\date{Dezembro, 2021}

\begin{document}

\maketitle
\pagebreak

\section{Sequências}

Define-se sequências numéricas como todo conjunto formado por meio de uma ``lógica'' por trás (semelhante à funções). A ``lógica'' é chamada de Lei de Formação, esta é apresentada de três formas diferentes.

\subsection{Tipos de Lei de Formação}

\subsubsection{Fórmula de Recorrência}

Nesse exemplo, a lei de formação é apresentada da seguinte forma: mostra-se inicialmente um método para identificar o primeiro número da sequência (por exemplo, $a_{1}$), e outro para calcular qualquer outro termo a partir desse $a_{1}$ pelo anterior $a_{n - 1}$

\begin{itemize}

  \item Escreva a sequência finita $f$ obedecendo à seguinte fórmula de recorrência: $a_{1} = 2$ e $a_{n - 1} + 3 \hspace{0.1cm} | \hspace{0.1cm} \forall n \in \{2, 3, 4\}$
\end{itemize}

Por hora, definido $a_{1} = 2$, concluí-se que $a_{2} = a_{1} + 3$ (lei de formação dada), portanto $a_{2} = 2 + 3 = 5$, executando para $a_{3}$ e $a_{4}$ obteremos o conjunto $f$ da sequência formada.

\[f = \{2, 5, 8, 11\}\]

\subsubsection{Termo em função de sua posição $n$}

Esse método permite ser direto na formulação da lei que rege a sequência. É dado uma fórmula que calcula $a_{n}$ em função de $n$ apenas (e sem a necessidade de conhecer $a_{1}$, como na forma anterior)

\begin{itemize}
    \item Escrever o conjunto $f$ seguindo $a_{n} = 2^{n}$ para $n = \{1, 2, 3\}$

        \begin{itemize}
                \item $2^{1} = 2$
                \item $2^{2} = 4$
                \item $2^{3} = 8$
                \item $f = \{2, 4, 8\}$
        \end{itemize}
\end{itemize}

\subsubsection{Propriedade da Sequência}

Esse modo consiste em escrever por extenso quais propriedades aquela sequência deverá ter, e montá-la a partir disso. Um exemplo dessa aplicação: ``Montar uma sequência infinita $f$ com os números primos na ordem crescente'' (já que não é possível escrever algo semelhante utilizando-se dos dois modos anteriores).

\section{Progressão Aritmética}

A progressão aritmética é um caso direto da aplicação da fórmula de recorrência apresentada na sessão anterior. Uma sequência do tipo P.A é montada a partir de duas afirmações.

\begin{equation*}
\begin{cases}

  a_{1} = a \\
  a_{n} = a_{n - 1} + r \\

\end{cases}
\end{equation*}

Logicamente, $n$ é um número real e maior ou igual a 2. Os elementos $a, r$ são números reais dados. Isso significa que, em uma sequência do tipo P.A, o próximo termo é igual ao termo anterior somado com uma constante $r$ (razão). \textbf{Esta razão pode ser obtida subtraíndo o próximo elemento com o anterior}.

\begin{itemize}
        \item $f = \{1, 3, 5, 7, 9, \dots\}$, onde $a_{1} = 1$ e $r = 2$ (notando sempre que o próximo elemento é a soma do anterior com 2)

        \item $f = \{-2, -4, -6, -8\}$, ($a_{1} = (-2)$ e $r = (-2)$)
\end{itemize}

\subsection{Classificação}

As classificações são simples e direta:

\begin{itemize}
        \item Crescente: Quando o próximo termo é maior que o anterior (ou seja, $r > 0$)
\end{itemize}

Essa condição só ocorre estritamente para $r$ positivo. Tomamos $f = \{a_{1}, a_{1} + r\}$, nesse contexto, afirmaremos

\[a_{1} + r > a_{1}\]
\[r > 0\]

\begin{itemize}
  \item Constante: $r = 0$
\end{itemize}

Nesse caso, não há uma variação por toda a sequência. Tomando a mesma sequência $f$ anterior, esperamos que o próximo termo seja igual ao primeiro:

\[a_{1} + r = a_{1}\]
\[r = 0\]

\begin{itemize}
  \item Descrescente: O próximo termo é menor que o anterior ($r < 0$), visualização análoga ao item crescente.
\end{itemize}

\subsection{Notações Úteis}

A fim de obter uma P.A ``forçadamente'' é interessante escrever a seguinte notação (no exemplo abaixo, pelo menos o 3 primeiros termos).

\[f = \{x - r,\hspace{0.1cm} x,\hspace{0.1cm} x + r\}\]
\[f = \{x,\hspace{0.1cm} x + r,\hspace{0.1cm} x + 2r\}\]

Uma questão exemplo envolvendo a manipulação da P.A deste modo.

\begin{itemize}
  \item Determine $a$ para que $\{a^{2},\hspace{0.1cm} (a + 1)^{2}\hspace{0.1cm}, (a + 5)^{2}\}$ seja uma P.A
\end{itemize}

Começaremos igualando o elemento em comum: a razão. Nesse modo, $r$ pode ser obtida subtraíndo $(a + 1)^{2}$ de $a^{2}$, mas também pode ser obtida pela subtração de $(a + 5)^{2}$ de $(a + 1)^{2}$.

\[(a + 1)^{2} - a^{2} = (a + 5)^{2} - (a + 1)^{2}\]

Desenvolvendo os termos

\[(a^{2} + 2a + 1) - a^{2} = (a^{2} + 10a + 25) - (a^{2} + 2a + 1)\]
\[2a + 1 = 12a + 26\]
\[10a = -25\]
\[a = -\frac{5}{2}\]

Portanto, $a$ deverá assumir o valor de $-5/2$ para que a sequência dada seja uma P.A, a partir da técnica ilustrada.
\end{document}
